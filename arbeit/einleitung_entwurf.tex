%\documentclass[12pt,a4paper]{scrartcl}
%
%\usepackage[ngerman]{babel}
%\usepackage[utf8]{inputenc}

%\begin{document}
\section*{Einleitung}
Unsere Lebenssituation zwischen Internat und heimischen Kinderzimmer hat uns in den vergangenen Jahren immer wieder vor technische Probleme gestellt. Immer war etwas auf dem Rechner abgespeichert, an dem man gerade nicht saß. Das konnten Texte füt die Schule sein, Fotos vom letzten besuchten Konzert oder schlicht die Frage: \glqq Habe ich den Rechner in der Schule/zu Hause ausgeschaltet?\grqq . Aus diesem Problem entwickelte sich unsere Idee für die Seminarfacharbeit: von einem Rechner auf einen anderen zugreifen zu können, um am Zielrechner Befehle auszuführen oder Dateienzu sichten.\\
Uns war bewusst, dass Daten speichern ud verwalten eine komplexe Aufgabe ist, die viele kleine Probleme aufwirft, für die wir eine Lösung finden wollten. So muss zum Beispiel genug Speicherplatz bereitgestellt werden und das Verschieben, Kopieren , Umbenennen, Auslesen oder Sichern der Daten ist ebenfalls nicht selbstverständlich. All das sollte nicht nur im Fall von uns Internatsschülern von jedem beliebigen Ort aus geschehen können, da viele Menschen nicht nur zu Hause auf z.B. Uralaubsbilder oder wichtige Dokumnte zugreifen wollen.\\
Die einfache Lösung bestünde darin, den Computer von einem anderen Ort aus fernzusteuern. Das ist aber entweder nur umständlich mit Funktionen des Betriebssystems zu realisieren, was bei verschiedenen Betriebssystemen sehr problematisch ist, oder mit schlecht erweiterbaren grafischen Systemen, die einen fortgeschrittenen Nutzer oft nicht zufrieden stellen.
Ein auf einer Ebene dazwischen angesiedeltes System zur Fernsteuerung scheint sinnvoll - also eine betriebssystemunabhängige Software zur Versendung von Steuerbefehlen und Dateien an einen über ein Computernetzwerk verbundenen Zielcomputer, welche dem Nutzer die Möglichkeit gibt, sie nach seinen spezifischen Ansprüchen selbständig zu erweitern.\\
Die Kommunikation mit dem Nutzer geschieht dabei über ein Terminal; die Eingabebefehle sind Teil einer selbst definierten formalen Sprache - damit werden unkomplizierte Bedienung und eine Möglichkeit zur Erweiterung kombiniert.
Der Nutzer soll in der Lage sein, den vorhandenen Kanon von Befehlen beliebig zu erweitern und mit eigenen Funktionen zu ergänzen, ohne das gesamte Programm neu kompilieren zu müssen. Deshalb wird die gesamte Eingabeverarbeitung intern von einer plattformunabhängigen Skriptsprache geregelt.
Dadurch soll der Nutzer außerdem in der Lage sein, das Programm mit eigenen Skripten auszubauen und Vorgänge im Programm je nach Bedarf zu automatisieren.\\
Auf der informationstechnischen Ebene ergeben sich mehrere Unterprobleme, deren Lösung wir anstreben. Als Grundlage soll ein Netzwerkprotokoll entwickelt werden.
Es sollte optimiert sein für die Übertragung von Anweisungen eines vom Nutzer erweiterbaren Befehlssatzes und Dateien jeglichen Typs. Dabei muss es sicherstellen, dass sowohl Dateien als auch Befehle übertragen werden können, und darüber hinaus die Daten sicher, fehlerfrei und im richtigen Format ankommen.\\
Zu guter Letzt muss das Programm modular aufgebaut sein, damit es von allen Entwicklern selbstständig entwickelt werden kann und niemand alle Details überblicken muss.\\
Insgesamt ist unser Ziel, den Anwender mit einer flexiblen Software bei der Fernsteuerung von Computern zu unterstützen, indem sich diese um Verbindungsaufbau, sowie Befehls- und Dateiübertragung kümmert und darüber hinaus eine Möglichkeit zur nutzerseitigen systematischen Erweiterung besteht.\\

Bei der Relisierung unseres Vorhabens erhielten wir wichtige Impulse und tatkräftige Unterstützung. Besonders Danken wir unserer Semirnarfachbetreuerin Frau Dr. M. Moor für die Beratung bei einer schwerwiegenden organisatorischen Problematik, unserem Fachbetreuer Herr J. Süpke für seine fachliche Unterstützung sowie das Mitwirken bei der Eingrenzung der Arbeitsziele und des Arbeitsinhalts und unserem Mitschüler Jan Sommerfeld für sein Teilhaben am Planungsprozess der Arbeit.  
%\end{document}
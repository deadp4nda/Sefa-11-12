\documentclass[12pt,a4paper]{scrartcl}

\usepackage[ngerman]{babel}
\usepackage[utf8]{inputenc}

\begin{document}
\section*{Einleitung}
Daten speichern und verwalten ist eine komplexe Aufgabe, die viele kleine Probleme beinhaltet. So zum Beispiel das Bereitstellen von genug Speicherplatz oder das Verschieben, Kopieren, Umbenennen, Auslesen oder Sichern der Daten; letzteres soll oft von jedem beliebigen Ort aus möglich sein, da man nicht nur zu Hause auf z.B. Urlaubsbilder, Musik und Videos zugreifen will; hat man viele\\
Die einfache Lösung würde darin bestehen, den Computer zu Hause fernzusteuern - das ist aber entweder umständlich mit Funktionen des Betriebssystems zu realisieren - was bei verschiedenen Betriebssystemen noch problematischer wird - oder mit schlecht erweiterbaren grafischen Systemen, die einen fortgeschrittenen Nutzer oft nicht zufrieden stellen.
Ein auf einer Ebene dazwischen angesiedeltes System zur Fernsteuerung wäre sinnvoll - also eine betriebssystemunabhängiges Software zur Versendung von Steuerbefehlen an einen über ein Computernetzwerk verbundenen Zielcomputer mit der Möglichkeit Dateien zu übertragen, welches den Nutzer in seinen Möglichkeiten nicht beeinträchtigt und trotzdem eine nicht überladene Oberfläche besitzt - das ist unsere Zielstellung.
%kp, mal irgendwann weiterschreiben

Dieses grob formulierte Ziel schlüsselt sich auf in mehrere Unterprobleme. Als Grundlage soll ein Netzwerkprotokoll entwickelt werden, welches unseren gestellten Ansprüchen gerecht erstellt werden muss. Es sollte also optimiert sein für die Übertragung von Anweisungen eines vom Nutzer erweiterbaren Satzes und Dateien jeglichen Typs; dabei muss es sicherstellen, dass die Daten sicher und fehlerfrei, sowie im richtigen Format ankommen.\\
Die Kommunikation mit dem Nutzer geschieht über ein Terminal; die Eingabebefehle sind Teil einer selbst definierten formalen Sprache - damit werden unkomplizierte Bedienung und eine Möglichkeit zur Erweiterung kombiniert.
Es soll dem Nutzer möglich sein, den vorhandenen Kanon an Befehlen und Argumenten beliebig zu erweitern, ohne das gesamte Programm neu kompilieren zu müssen. Deshalb wird die gesamte Eingabeverarbeitung intern von einer Implementierung durch eine Skriptsprache geregelt.\\
Zu guter Letzt muss unser Programm modular aufgebaut sein, damit es von allen drei Mitgliedern von den Schnittstellen abgesehen unabhängig entwickelt werden kann.
%meh. wird irgendwie nicht mehr xD
% ^+1 später... 

\end{document}
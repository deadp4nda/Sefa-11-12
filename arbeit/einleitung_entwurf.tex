\documentclass[12pt,a4paper]{scrartcl}

\usepackage[ngerman]{babel}
\usepackage[utf8]{inputenc}

\begin{document}
\section*{Einleitung}
Daten speichern und verwalten ist eine komplexe Aufgabe, die viele kleine Probleme beinhaltet. So zum Beispiel das bereitstellen von genug Speicherplatz oder das Verschieben, Kopieren, Umbenennen, Auslesen oder Sichern der Daten; letzteres soll oft von jedem beliebigen Ort aus möglich sein, da man nicht nur zu Hause auf z.B. Urlaubsbilder, Musik und Videos zugreifen will.\\
Die einfache Lösung würde darin bestehen, den Computer zu Hause fernzusteuern - das ist aber entweder umständlich mit Funktionen des Betriebssystems zu realisieren - was bei verschiedenen Betriebssystemen sogar noch problematischer wird - oder mit schlecht erweiterbaren grafischen Systemen, die einen fortgeschrittenen Nutzer oft nicht zufrieden stellen.
Ein auf einer Ebene dazwischen angesiedeltes System zur Fernsteuerung wäre sinnvoll - also ein Betriebssystemunabhängiges System zur Versendung von Steuerbefehlen an einen über ein Computernetzwerk verbundenen Zielcomputer mit der einfachen Möglichkeit Dateien zu übertragen, dass den Nutzer in seinen Möglichkeiten nicht beeinträchtigt und trotzdem eine einfache Oberfläche besitzt und eine relativ kleine Gesamtgröße - das ist unsere Zielstellung.
%kp, mal irgendwann weiterschreiben
\end{document}
\documentclass[12pt,a4paper]{scrartcl}

\usepackage[ngerman]{babel}
\usepackage[utf8]{inputenc}

\begin{document}
\section*{Einleitung}
Daten speichern und verwalten ist eine komplexe Aufgabe, die viele kleine Probleme beinhaltet. So zum Beispiel das Bereitstellen von genug Speicherplatz oder das Verschieben, Kopieren, Umbenennen, Auslesen oder Sichern der Daten; letzteres soll oft von jedem beliebigen Ort aus möglich sein, da man nicht nur zu Hause auf z.B. Urlaubsbilder, Musik und Videos zugreifen will.\\
Die einfache Lösung würde darin bestehen, den Computer zu Hause fernzusteuern - das ist aber entweder umständlich mit Funktionen des Betriebssystems zu realisieren - was bei verschiedenen Betriebssystemen noch problematischer wird - oder mit schlecht erweiterbaren grafischen Systemen, die einen fortgeschrittenen Nutzer oft nicht zufrieden stellen.
Ein auf einer Ebene dazwischen angesiedeltes System zur Fernsteuerung wäre sinnvoll - also eine betriebssystemunabhängiges Software zur Versendung von Steuerbefehlen an einen über ein Computernetzwerk verbundenen Zielcomputer mit der Möglichkeit Dateien zu übertragen, welches den Nutzer in seinen Möglichkeiten nicht beeinträchtigt und trotzdem eine intuitive Oberfläche besitzt - das ist unsere Zielstellung.
%kp, mal irgendwann weiterschreiben


DIeses grob formulierte Ziel schlüsselt sich auf in mehre Unterprobleme. Als Grundlage soll ein Netzwerkprotokoll entwickelt werden, welches mithilfe der Methodik des Protokolldesigns unseren Ansprüchen gerecht werden soll. Es sollte also optimiert sein für ein Übertragung von Anweisungen und Datein jeglichen Typs.\\ 
Zur Kommunikation zwischen dem Benutzer und unserem Programm soll als Syntax eine formale Sprache modelliert und implementiert werden, welche eine Anpassung an die Bedfürnisse des Nutzers ermöglichen soll.\\
Zu guter letzt muss unser Programm Implementert werden, wobei die Mitarbeit mehrerer Gruppenmitglieder eine strikt organisierte Arbeitsweise in der Planung, Aufgabenverteilung und Fehlerbehebung benötigt.  
%meh. wird irgendwie nicht mehr xD

\end{document}
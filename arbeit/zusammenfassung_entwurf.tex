\documentclass[12pt,a4paper]{scrartcl}

\usepackage[ngerman]{babel}
\usepackage[utf8]{inputenc}

\begin{document}
\section*{Zusammenfassung}
Unsere grundlegende Zielstellung der Arbeit forderte es, ein Programm zu entwickeln und implementieren, welches die Möglichkeiten bereitstellt, sowohl Anweisungen als auch Datein über eine bestehende Netzwerkverbindung zwischen zwei Computern zu verschicken.\\
Zur Verwirklichung dieses Ziels haben wir uns mit dem Design eines eigenen Netzwerkprotokolls beschäftigt, welches die ablaufenden Kommunikationsprozesse zwischen den zwei Rechnern definiert. Es baut auf dem gängigen Protokollstandart TCP/IP auf und gliedert die zu übertragenen Daten in eine für unsere Anwendung optimierte Form. 
Für die Eingabe der Wünsche der Benutzer entwickelten wir eine eigene Eingabesyntax auf der Basis einer Formalen Sprache. Im Gegensatz zu natürlichen Sprachen kennzeichen sich diese grundsätzlich durch eine Eindeutigkeit, ausgehend von einer klar definierten Grammatik. Dies ermöglicht eine problemlose, automatisierte Interpretation der Nutzereingaben durch das Programm.
Neben den zwei größeren Unterzielen prägten auch mehre kleinere Faktoren unsere Entwicklung. So sollte das Endprodukt möglichst universell anwendbar sein, also beispielsweise auf jegliche Datentypen anwendbar sein, sowie auf unterschiedlichen Betriebssytemarchitekturen lauffähig sein. Außerdem stellt die Erweiterbarkeit und Anpassungsmöglichkeit an die Ansprüche unterschiedlicher Benutzer ein Prinzip da, nach welchem wir unsere Entwicklung richteten. Sowohl das Protokoll als auch die Eingabesyntax sind so implementiert, dass durch geringen Aufwand neue Programmfunktionen eingegliedert werden können.\\
Eine in unseren Augen sinnvolle Weiterführung der Arbeit wäre die Portierung der Software auf mobile Geräte, wie zum Beispiel Handys und Tablets, da diese stetig voranschreitend die klassischen Computer in alltäglichen Aufgaben ersetzen. Außerdem würde eine solche Erweiterung unsere Anfängliche Motivation, einen Computer unabhängig von seiner physischen Position steuern zu können, weiter Umsetzen, da es eine Fernsteuerung von überall ausgehend ermöglicht. Besonderes die Benutzeroberfläche betreffend, sind einige kleine Erweiterungen sinnvoll, welche sich positiv auf die Bedienerfreundlichkeit auswirken würden. So ließe sich beispielsweise eine mächtigere Konsole entwickeln, welche durch Features wie die automatische Vervollständigung von Eingaben, die Möglichkeit vorherige Anweisungen zu wiederholen und das Erlauben von Drag’n’Drop für Datein, dem Benutzer die Arbeit mit unserem Programm erleichtert.\\ 
Unser zu Beginn harmonisches Gruppen- und Arbeitsklima wurde kurz vor der regulären Abgabe durch den Ausfall eines Gruppenmitglieds drastisch beeinflusst. Da die Kommunikation mit besagtem Gruppenmitglied nach mehreren Anläufen fehlschlug, und der vereinbarte Arbeitsanteil nicht erbracht wurde, war eine spontane Neuverteilung der Aufgaben zwischen den verbleibenden Gruppenmitglieder nötig. Die ab diesem Punkt weiterführende Arbeit war geprägt durch eine hohe Produktivität, resultierend aus einer starken Kommunikation. Gestüzt durch die Projektentwicklungsplattform GitHub entstand eine dynamische Zusammenarbeit mit der Möglichekeit spontan Aufgabenbereiche umzuverteilen, so dass die Arbeit trotz veringerter Gruppengröße erfolgreich fertig gestellt werden konnte.

\end{document}
%\documentclass[12pt,a4paper]{scrartcl}
%
%\usepackage[ngerman]{babel}
%\usepackage[utf8]{inputenc}

%\begin{document}
\section{Zusammenfassung}
Unsere grundlegende Zielstellung der Arbeit war es ein Programm zu entwickeln und zu implementieren, welches die Möglichkeiten bereitstellt sowohl Anweisungen als auch Dateien über ein Netzwerk zwischen zwei Computern zu verschicken.\\\\
Zur Verwirklichung dieses Ziels haben wir uns mit dem Design eines eigenen Netzwerkprotokolls beschäftigt, welches die ablaufenden Kommunikationsprozesse zwischen den zwei Rechnern definiert. 
Es baut auf dem gängigen Protokollstandard TLS auf und gliedert die zu übertragenden Daten in eine für unsere Anwendung optimierte Form. 
Zur Nutzereingabe entwickelten wir eine eigene Eingabesyntax auf der Basis einer selbst entwickelten formalen Sprache. 
Im Gegensatz zu natürlichen Sprachen kennzeichnen sich diese grundsätzlich durch ihre Eindeutigkeit, ausgehend von einer klar definierten Grammatik. 
Dies ermöglicht eine problemlose, automatisierte Interpretation der Nutzereingaben durch das Programm und damit eine reibungslose Kommunikation mit einem Benutzer.
Neben den zwei größeren Unterzielen prägten auch mehre kleinere Zielstellungen unsere Entwicklung.
So sollte das Endprodukt möglichst universell anwendbar sein, also beispielsweise auf jegliche Datentypen, und es sollte auf unterschiedlichen Betriebssytemarchitekturen lauffähig sein, was beides mit der entwickelten Software erfüllt ist.
Außerdem stellt die Erweiterbarkeit und Anpassungsmöglichkeit an die Ansprüche unterschiedlicher Benutzer ein Prinzip da, nach welchem wir unsere Entwicklung richteten; deshalb sind sowohl das Protokoll als auch die Eingabesyntax so implementiert, dass durch geringen Aufwand neue Programmfunktionen eingegliedert werden können, ohne das Programm neu kompilieren zu müssen.\\\\
Eine in unseren Augen sinnvolle Weiterführung der Arbeit wäre die Portierung der Software auf mobile Geräte, wie Handys und Tablets, da diese stetig voranschreitend die klassischen Computer bei alltäglichen Aufgaben ersetzen und größere Mobilität gewährleisten. 
So würde eine solche Erweiterung unsere anfängliche Motivation, einen Computer unabhängig von seiner physischen Position steuern zu können, weiter auslegen, da es eine Fernsteuerung von überall her ermöglicht. 
Die Benutzeroberfläche betreffend sind einige kleine Erweiterungen sinnvoll, welche sich positiv auf die Bedienerfreundlichkeit auswirken würden. 
So ließe sich beispielsweise eine mächtigere Konsole entwickeln, welche durch Features wie die automatische Vervollständigung von Eingaben, das schreiben von Batch-ähnlichen Skripten oder das Erlauben von Drag’n’Drop für Dateien dem Benutzer die Arbeit mit unserem Programm erleichtert.
Als Funktionserweiterung für das Protokoll wäre in Zukunft eine Auslegung auf Streaming möglich, mit der zum Beispiel eine grafische Übertragung oder das Ansehen einer Videodatei ohne vorheriges Übertragen möglich wäre.\\\\
Unser zu Beginn harmonisches Gruppen- und Arbeitsklima wurde kurz vor der regulären Abgabe durch den Ausfall eines Gruppenmitglieds drastisch beeinflusst. 
Da die Kommunikation mit besagtem Gruppenmitglied nach mehreren Anläufen fehlschlug, und der vereinbarte Arbeitsanteil nicht erbracht wurde, war eine spontane Neuverteilung der Aufgaben zwischen den verbleibenden Gruppenmitglieder nötig. 
Die ab diesem Punkt weiterführende Arbeit war geprägt durch eine hohe Produktivität, resultierend aus einer starken Kommunikation.
Gestützt durch die Projektentwicklungsplattform GitHub entstand eine dynamische Zusammenarbeit mit der Möglichkeit spontan Aufgabenbereiche umzuverteilen, so dass die Arbeit trotz verringerter Gruppengröße erfolgreich fertig gestellt werden konnte.
%\end{document}
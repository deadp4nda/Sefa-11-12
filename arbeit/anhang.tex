\section*{Anhang}
\subsection*{Glossar}

\begin{table}[h]
\begin{tabularx}{\textwidth}{l X}
\glsref{Header}: &  Ein Datensatz der an den Anfang eines im Netzwerk verschickten Paketes geschrieben wird um dessen sichere Ankunft sicherzustellen oder ein Teil des Quelltextes, in dem bestimmte Datentypen und Strukturen definiert aber nicht deklariert ist.\\
\glsref{Paket, Netzwerkpaket}: & Informationseinheit, die von Computer zu Computer in einem Netzwerk versendet wird\\
\glsref{Protokoll, Netzwerkprotokoll}: & Spezifikation eines Teils des Headers eines Pakets; kann außerdem auch einen Ablauf für die Kommunikation zwischen den an der Verbindung beteiligten Rechnern beinhalten\\
\glsref{ISO/OSI Modell} & Ein informatisches Modell zur Darstellung der Kommunikation in einem Netzwerk. Dabei werden verschiedene "`Schichten"' definiert, die jeweils unterschiedliche Aufgaben bei der Kommunikation übernehmen und aufeinander aufbauen. So gibt es die unteren, noch sehr hardwarenahen  Schichten, die sich um die korrekte Übertragung von einzelnen Bits kümmern; Schichten darüber, welche für die Ankunft einzelner \glsref{Paket}e am richtigen Ort zuständig sind bis hin zu Schichten, die Abläufe und Formate enthalten, die für die Darstellung und Übertragung von Websites zuständig sind.\\
\end{tabularx}
\end{table}
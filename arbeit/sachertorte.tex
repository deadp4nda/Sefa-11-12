\section{Benutzeroberfläche und Funktionsumfang des Programms}
\subsection{Benutzeroberfläche und Bedienung}

Das Programm präsentiert sich dem Nutzer als selbstentwickeltes Terminal(s. Abb. \ref{blah}). %TODO bild einfügen
Damit ist es möglich, in Textform Eingaben zu machen, die dann vom Eingabeverarbeitungsteil interpretiert und ausgeführt werden, sowie Rückmeldungen vom Programm zu erhalten.
So kombiniert sich eine einfache und für Konsolenbenutzer intuitive Oberfläche, die den Nutzer nahezu überhaupt nicht in seinen Möglichkeiten einschränkt.
Die Kommunikation mit dem Programm erfolgt über einen definierten, aber erweiterbaren Befehlssatz samt Argumenten - eine genaue Erläuterung des Befehlssatzes ist in Kapitel X:X:X anzufinden[s. S. \pageref{blahkeineahnung}]. %TODO Leo, bitte mach was deswegen
\subsection{Funktionsumfang und dessen Nutzergestützte Erweiterung}
Die Software selbst bietet die Möglichkeit, Dateien ohne weitere Konfiguration in beide Richtungen zu übertragen, auszuführen oder anzuzeigen.
Unabhängig von Dateien gibt es natürlich noch Befehle zur generellen Bedienung des Programms.
Das Prinzip der geringen Einschränkung zieht sich weiter durch das Projekt - so ist es möglich, Konsolenbefehle an den Zielcomputer zu senden und die Ausgabe zu erhalten.
Dafür ist ein Verständnis der Funktionen nötig, die ausgeführt werden, und daher ist es ein Bestandteil für etwas fortgeschrittenere Nutzer; dafür ist damit eine echte Fernsteuerung des Zielcomputers möglich.
Durch diesen Aufbau ist das Programm, wie es geplant war eine Unterstützung für etwas erfahrenere Nutzer, die aber nicht zur Fernsteuerung auf umständliche Funktionen des Betriebssystems zugreifen wollen, um einen Computer über ein Netzwerk zu steuern und die ein Funktionsspektrum haben wollen, das so breit wie möglich ist.\\
Dazu gibt es die Möglichkeit einer nutzergestützten Erweiterung der Software, was weiterhin "`modding"' genannt wird.
Um ein möglichst breites Spektrum an Nutzern anzusprechen, kann ein Nutzer das Programm selbst zu Teilen mitgestalten, indem er eigene Befehle definiert und eine Funktion bereitstellt, die die Eingabeverarbeitung des Befehls übernimmt.
Das kann als Luascript eingebunden werden, und erfordert somit keine Neukompilierung des ganzen Projekts.
Diese Funktion kann dann irgendeine beliebige Aktion ausführen, die nicht einmal unbedingt nur mit dem Programm zu tun haben muss - dadurch wird es ermöglicht viele Abläufe zu automatisieren und auf Basis des Programms theoretisch eine Systemverwaltungssoftware nach Bedarf erstellt werden kann.
Die Sicherheitslücke, die dadurch entsteht ist vergleichsweise äußerst gering, da hier nicht von Fremden auf fremde Systeme zugegriffen wird, sondern nur bekannte Systeme verbunden werden.
Zusätzlich dazu gibt es auf der Zielseite die Notwendigkeit einer Authentifizierung, wodurch nicht einmal mit schlechter Absicht eine ungewollte Fernsteuerung möglich ist.
Dafür gibt es auf Basis grundlegender, einfacher Funktionen die Möglichkeit einer beliebig großen Erweiterung.
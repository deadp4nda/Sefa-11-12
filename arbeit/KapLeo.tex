\documentclass[12pt, a4paper]{scrartcl}
\usepackage[german]{babel}
\usepackage[utf8]{inputenc}
\usepackage{graphicx}

\begin{document}
\section{Funktionsumfang und Eingabeverarbeitung auf der Anwendungsebene}

\subsection{Entwicklung einer eigenen Scriptsprache zur Anwendungssteuerung}
\subsubsection{Theoretische Grundlagen formaler Sprachen}
Wir als Menschen haben die gesprochenen Sprachen entwickelt, um uns untereinander zu verständigen und Informationen auszutauschen. Um Misverständnisse zu vermeiden, ist unsere Kommunikation durch strikte grammatikalische und semantische Regeln definiert. Es werden strukturelle Merkmale wie zum Beispiel die Reihenfolge einzelner Sprachteile festgelegt, außerdem gibt es einen begrenzten Wortschatz, indem Jedes Wort eine bestimmte Bedeutung hat.
Um auch Computern Informationen zu übermitteln, um eine maschinelle Verarbeitung dieser zu ermöglichen, benutzt man formale Sprachen. Diese gleichen von den grundlegenden strukturellen Ansprüchen der natürlichen Sprachen, verfügen also auch über eine Syntax und eine Semantik, definieren sich allerdings dadurch, dass diese für einen Rechner eindeutig verständlich sein müssen, und über keinen spracheliche Mittel verfügen, welche Interpretationsfreiraum lassen. Ziel ist es klare Anweisungen in einer für den Computer verständlichen Weise zu übermitteln.\\
Da Sprachen in der Regel durch das Aneinandereihen von einzelnen Sprachelementen über eine Unendlichkeit verfügen, nutzt man künstliche Grammatiken der Form $G = (N, T, P, s)$, um diese zu beschreiben. Eine Grammatik $G$ dieser Art verfügt über eine Menge von Nichtterminalen $N$, oder auch Variablen genannt, welche im Verlauf der Wortbildung durch die Menge der Produktionsregeln $P$ der Form $U \rightarrow V$ durch eine Kombination aus Terminalen der Menge $T$ ersetzt werden. Das Startsymbol $s$ ist zwingend die erste Nichtterminale, welche durch eine Produktion der Form $s \rightarrow V$ ersetzt wird. Ein Wort gehört einer Grammatik $G$ an, wenn es nurnoch aus Terminalen Symbolen der Menge $T_G$ besteht und ausgehend von dem Startsymbol $S_G$ mit den Produktionen der Menge $P_G$ gebildet werden kann.\footnote{$	https://www.uni-ulm.de/fileadmin/website_uni_ulm/iui.inst.040/Formale_Methoden_der_Informatik$\\$/Vorlesungsskripte/FMdI-06--2010-01-10--FormaleSprachen_Vorlesung.pdf$
; 19.11.2018}\\
Je nachdem, nach welcher Art die Produktionsregeln aufgebaut sind, lässt sich eine formale Sprache nach dem Informatiker Noam Chomsky in vier Typen einteilen. Grundsätzlich gehört jede Sprache dem Typ 0 der allgemeinen Sprachen an. Eine Sprache gehört immer einem Typ $X$ an, wenn alle Bedingungen  der Typen $\leq X$ für alle Produktionsregeln der Form $U \rightarrow V$ erfüllt sind. Die Bedingung für eine kontextsensitive Sprache des Typs 1 besagt, dass durch eine Produktionsregel das Wort nicht verkürt werden kann, als $|U| \leq |V|$. Für eine kontextfreie Sprache des Typs 2 gilt, dass durch eine Produktionsregel nur jeweils ein Nichtterminal ersetzt werden kann, also $|U| = 1$. Eine Sprache ist eine reguläre Sprache des Typs 3, wenn ein Nichtterminal durch eine Produktionsregel entweder durch ein Terminal oder durch ein Terminal und ein Nichtterminal ersetzt wird, jedoch nicht durch mehre Nichtterminale, also $U \leq a|aB$. \footnote{I. Wegener;	Theoretische Informatik; Kap.5}\\
Um die Darstelung von Sprachen zu vereinheitlichen, wurde die Erweiterte-Backus-Nauer-Form, kurz EBNF entwickelt. Die oben gegebenen Zeichen sind Teil des EBNF Iso Standards und durch sie lässt sich jede beliebige Formale Sprache beschreiben.\\
\begin{table}[]
\begin{tabular}{|ll|ll|}
\hline
Verwendung                    & Zeichen   & Verwendung                    & Zeichen \\ \hline
Definition                    & =         & Aufzählung                    & ,       \\
Endezeichen                   & ;         & Alternative                   & |       \\
Option                        & {[}...{]} & Optionale Wiederholung        & \{...\} \\
Gruppierung                   & (....)    & Anführungszeichen 1. Variante & "..."   \\
Anführungszeichen 2. Variante & '...'     & Kommntar                      & (*...*) \\
Spezielle Sequenz             & ?...?     & Ausnahme                      & -       \\ \hline
\end{tabular}
\end{table}
 
\subsubsection{Spezifikation unserer Scriptsprache anhand unserer Ansprüche an den Funktionsumfang}

\subsection{Implementierung der Anwendungssteuerung in der Scriptsprache Lua}

\subsubsection{Darstellung des grundlegenden Programmaufbaus und Programmablaufs}
Grundlegende Aufgabe der Anwendungssteuerung ist es, die Eingaben die der Nutzer unserer Software tätigt, in Aktionen des Programms umzusetzten. Dazu werden zum einen auf dem Rechner der sendenden Person die, nach der in Kaptiel x.1 beschriebenen Syntax, eingegebene Anweisung in der Eingabeverarbeitung auf ihre Korrektheit überprüft, um anschließend aufgeschlüsselt zu werden und die gewünschte Sendefunktion zu initialisieren.\\
Nachdem die Daten mithilfe des Netzwerkprotokolls übermittelt wurden, werden sie von der Übertragungsverarbeitung, auf dem Empfängercomputer weiter bearbeitet. Je nach spezifiziertem Typ der Übermittlung, wird entweder die enthaltene Datei auf Fehler überprüft und anschließend endgültig gespeichert, oder die auszuführende Anweisung wird an die betirebsystemeigene Kommandozeile weitergegeben und ausgeführt.\\
Um keine ungewollten Übertragungen von potenziell schädlichen Dateien oder Anweisungen zu ermöglichen, wird für die Ausführung jeder Übermittlungsanfrage die Bestätigung des Empfängers benötigt.\\\hfill\\
\includegraphics[scale=.45]{anw.jpg}
\subsubsection{Ansprüche an die Implementierung und die daraus resultierende Wahl der Programmiersprache}
Wie in Kapitel 3.2.1 erläutert, wird auf der Anwendungsebene unseres Programmes, also dem Teil welcher ausschließlich auf einem Rechner läuft, die grundlegende Funktionalität implementiert. Da unser selbstentwickeltes Netztwerkprotokoll, welches die eigendliche Kommunikation zwischen 2 Computern bewirkt, relativ universell auf verschiedenartige Daten anwendbar ist, stellt dieses im Bezug auf den Funktionsumfang der Software nicht den limitierenden Faktor da. Um das Protokoll eventuell auch im Zeitraum nach der eigendlichen Arbeit ausnutzen zu können, und nicht durch einen festgelegten Satz von Befehlen beschränkt zu werden, entschieden wir uns dazu, diese Softwareebene in einer Scriptsprache zu implementieren.\\
Diese Klasse von Programmiersprachen kennzeichnet sich dadurch aus, dass der vom Menschen lesbare Quelltext eines Programmes erst bei seiner Ausführung in Anweisungen übersetzt werden, welche für den Computer ausführbar sind. Im gegensatzt zu kompilierten Sprachen hat dies den Vorteil, dass Programmcode unkompliziert korrigiert und modifiziert werden kann.\\
In diesem Zusammenhang ergibt sich außerdem eine leichte Wartbarkeit, also die Möglichkeit eventuell auftretende Fehler zu beseitigen, als ein weiterer erfüllter Anspruch. Da wir stets zu dritt an unserer Software programmiert haben, sollte es jedem von uns mit möglichst geringem Aufwand möglich sein, Programmteile zu verbessern, auch wenn es sich nicht um den ursprünglichen Autor handelt.\\
Den Vorteil von kompilierten Programmiersprachen, dass sie sich durch eine höhere Leistungsfähigkeit besonders bei komplexeren Problemen auszeichnen, haben wir uns durch die Implementierung unserer Netzwerkteils in der Sprache C++ zu nutzen gemacht. Dies stellt den Anspruch, dass die gewählte Scriptsprache problemlos in Software der Sprache C++ einzubetten ist, und keine Probleme bei der Kommunikation zwischen Programmen der zwei Sprachen auftreten.\\
Da wir als Gruppenmitglieder unterschiedliche Betriebsysteme auf unseren Arbeitsrechnern benutzen, entstand als Nebenprodukt die Anforderung, dass das finale Programm systemunabhängig sein muss, also ohne erheblichen Aufwand auf neue Betriebssysteme portierbar ist. Unser Fokus lag dabei für die Arbeit auf Computersystemen und wir haben mobile Geräte vorerst außen vor gelassen.
Zu guter Letzt ist der Zeitraum der Seminarfacharbeit auch nur auf andertalb Jahre begrenzt, weshalb die gewählte Sprache mit einem geringem Lernaufwand benutzbar sein muss.\\
Als Kompromiss zwischen allen beschriebenen Ansprüchen entschieden wir uns für die Programmiersprache Lua. Da sie in reinem C geschrieben ist, weißt sie eine uneingeschränkte Integrationmöglichkeit unseres Netzwerkprotokolls vor, und lässt sich außerdem auf die meist verbreitesten Betriebssytemarchitekturen wie Windows und Unix portieren.\footnote{www.lua.org/about.html, 17.11.2018} Ein positiver Nebeneffekt dieser Wahl ist die hohe Leistungsfähigkeit von Lua, verglichen zu anderen weit verbreiteten Scriptsprachen. Der Sprachaufbau orientiert sich sehr nah an der Englischen Srache, was ein schnelles Erlernen ermöglichte. 


\subsubsection{Erläuterung wesentlicher Elemente der Implementierung}


\end{document}
